\documentclass[a4paper,twoside,11pt]{article}    % base article class
\usepackage[utf8]{inputenc}         % allow utf8 input
%\usepackage{textalpha}              % enable greek utf8 in text
%\usepackage{alphabeta}              % enable greek utf8 in math
\usepackage{graphicx}               % includegraphics
\usepackage{amsmath,amsthm}         % fancier math
%
%\usepackage{ebgaramond}
%
%\usepackage{newtxtext}               % times new roman font
%\usepackage[bigdelims,cmintegrals,vvarbb]{newtxmath}
%
\usepackage[osf,sups]{Baskervaldx} % lining figures
\usepackage[bigdelims,cmintegrals,vvarbb,baskervaldx,frenchmath,upint]{newtxmath} % math font
%\usepackage{cabin}                  % monospace font
\usepackage[cal=boondoxo]{mathalfa} % mathcal from STIX, unslanted a bit
%
\usepackage{hyperref,url}           % hyperlinks and urls
\usepackage{fancyvrb}               % fancier verbatim environment
\usepackage[x11names]{xcolor}                 % for colors
\usepackage{geometry}              % to change page margins
%\usepackage{ulem}              % for striketrhough
%\usepackage{pdflscape}              % for landscape pages
\usepackage[vlined,algoruled]{algorithm2e} % to typeset floating algorithms
%\delimitershortfall-1sp
%\usepackage{mleftright}              % enlarge nested parentheses
%\mleftright
\setlength{\parindent}{0pt}         % no paragraph indentation
\setlength{\parskip}{7pt}           % spacing between paragraphs
\setlength{\emergencystretch}{3em}  % prevent overfull lines
\pdfimageresolution 200             % change the default for included png files
\pdfinfoomitdate=1\pdftrailerid{}   % ensure reproducible PDF
\makeatletter\renewcommand{\verbatim@font}{\ttfamily\footnotesize}\makeatother
\newenvironment{gallery}{}{}
\newcommand{\galleryline}[1]{\includegraphics{#1}{\footnotesize\tt #1}\newline}
\theoremstyle{note}
\newtheorem{exercice}{Exercice}
\swapnumbers
\theoremstyle{plain}
\newtheorem{theorem}{Theorem}[section]
\newtheorem{lemma}[theorem]{Lemma}
\newtheorem{remark}[theorem]{Remark}
\newtheorem{definition}[theorem]{Definition}
\newtheorem{proposition}[theorem]{Proposition}

\newcommand{\1}{\mathbf{1}}
\newcommand{\Z}{\mathbf{Z}}
\newcommand{\N}{\mathbf{N}}
\newcommand{\Q}{\mathbf{Q}}
\newcommand{\R}{\mathbf{R}}
\newcommand{\C}{\mathbf{C}}
\newcommand{\T}{\mathbf{T}}
\newcommand{\PP}{\mathcal{P}}
\newcommand{\PN}{\mathbf{P}}
\newcommand{\ud}{\mathrm{d}}
\newcommand{\ds}{\displaystyle}
\newcommand{\DFT}{\mathtt{DFT}}
\newcommand{\IDFT}{\mathtt{IDFT}}
\newcommand{\x}{\mathbf{x}}
\newcommand{\y}{\mathbf{y}}
\newcommand{\z}{\mathbf{z}}
\newcommand{\q}{\mathbf{q}}
%\newcommand{\u}{\mathbf{u}}
\newcommand{\n}{\mathbf{n}}
\renewcommand{\Re}{\operatorname{Re}}
\renewcommand{\Im}{\operatorname{Im}}
\newcommand{\abs}[1]{\left|#1\right|}
\newcommand{\Abs}[1]{\left\|#1\right\|}
\newcommand{\parens}[1]{\left(#1\right)}
\newcommand{\pairing}[2]{\left\langle #1,#2\right\rangle}


% widecheck
% (code from mathabx.sty and mathabx.dcl)
\DeclareFontFamily{U}{mathx}{\hyphenchar\font45}
\DeclareFontShape{U}{mathx}{m}{n}{
	<5> <6> <7> <8> <9> <10>
	<10.95> <12> <14.4> <17.28> <20.74> <24.88>
	mathx10
}{}
\DeclareSymbolFont{mathx}{U}{mathx}{m}{n}
\DeclareFontSubstitution{U}{mathx}{m}{n}
\DeclareMathAccent{\widecheck}{0}{mathx}{"71}
\DeclareMathAccent{\wideparen}{0}{mathx}{"75}
% really wide hat
\usepackage{scalerel,stackengine}
\stackMath
\newcommand\reallywidehat[1]{%
\savestack{\tmpbox}{\stretchto{%
  \scaleto{%
    \scalerel*[\widthof{\ensuremath{#1}}]{\kern-.6pt\bigwedge\kern-.6pt}%
    {\rule[-\textheight/2]{1ex}{\textheight}}%WIDTH-LIMITED BIG WEDGE
  }{\textheight}% 
}{0.5ex}}%
\stackon[1pt]{#1}{\tmpbox}%
}

\begin{document}
{\Large
\begin{center}
	Comprehensive evaluation of all Gaussian blur implementations
\end{center}
}

\begin{abstract}
We propose an unified interface to run all freely available Gaussian blur
implementations.  We use this interface to compare and evaluate the results 
\end{abstract}

\section{Theory and practice of Gaussian blurring}

\subsection{The Gaussian function in~$\R^d$}

The {\bf Gaussian function} in Euclidean~$d$-dimensional space is defined as
\[
	G_{\sigma}(\x)=\frac1{\left(\sigma\sqrt{2\pi}\right)^d}\exp\frac{-\Abs{\x}^2}{2\sigma^2}.
\]
for~$\sigma>0$.
It is a smooth, strictly positive, radial function of integral 1.

This function is~{\bf separable} along the dimensions, meaning
that~$G_\sigma((\x,\y))=G_\sigma(\x)G_\sigma(\y)$
for~$(\x,\y)\in\R^p\times\R^{d-p}$.  Or more
generally~$G_\sigma(\x)=\prod_iG_\sigma(x_i)$.

The {\bf derivatives} of~$G$ can be computed by elementary means:
\[
	\frac{\partial G_\sigma(\x)}{\partial x_i}
	=
	\left(-\frac{x_i}{\sigma^2}\right)
	G_\sigma(\x)
	\qquad
	\qquad
	\frac{\partial^2 G_\sigma(\x)}{\partial x_i^2}
	=
	\left(\frac{x_i^2}{\sigma^4}-\frac1{\sigma^2}\right)
	G_\sigma(\x)
	\qquad
	\qquad
	\Delta G_\sigma(\x)
	=
	\left(\frac{\Abs{\x}^2}{\sigma^4}-\frac d{\sigma^2}\right)
	G_\sigma(\x)
\]
from this last expression we see that the laplacian changes sign at a
radius~$\Abs{\x}=\sigma\sqrt{d}$, separating a convex behavior around the
origin, to a concave region in the rest of space.
The partial derivative with respect to~$\sigma$
\[
	\frac{\partial G_\sigma(\x)}{\partial\sigma}
	=
	\left(\frac{\Abs{\x}^2}{\sigma^3}-\frac d{\sigma}\right)
	G_\sigma(\x)
\]
looks similar to the Laplacian.  We could say that it satisfies a heat
equation with time-dependent
coefficient~$\frac{\partial}{\partial\sigma}G=\sigma\Delta G$.  We can obtain
an actual {\bf heat equation} by re-parametrizing~$t=2\sigma^2$ so
that~$\frac{\partial}{\partial t}G=\Delta G$.

Finally, {\bf Fourier transform} of~$G_\sigma$ is
\[
	\widehat{G_\sigma}(\y)=\exp\frac{-\sigma^2\Abs{y}^2}2.
\]

\subsection{Gaussian blur in $\R^2$}

{\bf Gaussian blur} is defined as convolution by the Gaussian function.  

We are interested in the particular case~$d=2$
\[
	G_{\sigma}(x,y)=\frac1{2\pi\sigma^2}\exp\frac{-(x^2+y^2)}{2\sigma^2}.
\]

define scale space

Let~$f:\R^2\to\R$ be a suitable function, we denote by~$G_\sigma f$ the
function
\[
	G_\sigma f (x) := \parens{G_\sigma * f}(x)
\]
Applying $G_\sigma$ is thus called~\emph{blurring the function by a Gaussian
of size~$\sigma$}.  More generally, the three-variable function
\[
	(\sigma,x,y)\mapsto G_\sigma f(x,y)
\]
is called the~{\bf Gaussian scale space} of~$f$.


\subsection{Gaussian blur in $\T^2$}

\subsection{Discrete Gaussian blur}

\subsection{Implementation strategies}



\section{Overview of available implementations}



\section{Unified interface}



\section{Evaluation}

\end{document}
% vim:set tw=77 filetype=tex spell spelllang=en:
